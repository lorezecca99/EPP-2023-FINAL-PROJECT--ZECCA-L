\documentclass[11pt, a4paper, leqno]{article}
\usepackage{a4wide}
\usepackage[T1]{fontenc}
\usepackage[utf8]{inputenc}
\usepackage{float, afterpage, rotating, graphicx}
\usepackage{epstopdf}
\usepackage{longtable, booktabs, tabularx}
\usepackage{fancyvrb, moreverb, relsize}
\usepackage{eurosym, calc}
% \usepackage{chngcntr}
\usepackage{amsmath, amssymb, amsfonts, amsthm, bm}
\usepackage{caption}
\usepackage{mdwlist}
\usepackage{xfrac}
\usepackage{setspace}
\usepackage[dvipsnames]{xcolor}
\usepackage{subcaption}
\usepackage{minibox}
% \usepackage{pdf14} % Enable for Manuscriptcentral -- can't handle pdf 1.5
% \usepackage{endfloat} % Enable to move tables / figures to the end. Useful for some
% submissions.
\usepackage[
    natbib=true,
    bibencoding=inputenc,
    bibstyle=authoryear-ibid,
    citestyle=authoryear-comp,
    maxcitenames=3,
    maxbibnames=10,
    useprefix=false,
    sortcites=true,
    backend=biber
]{biblatex}
\AtBeginDocument{\toggletrue{blx@useprefix}}
\AtBeginBibliography{\togglefalse{blx@useprefix}}
\setlength{\bibitemsep}{1.5ex}
\addbibresource{../../paper/refs.bib}

\usepackage[unicode=true]{hyperref}
\hypersetup{
    colorlinks=true,
    linkcolor=black,
    anchorcolor=black,
    citecolor=NavyBlue,
    filecolor=black,
    menucolor=black,
    runcolor=black,
    urlcolor=NavyBlue
}


\widowpenalty=10000
\clubpenalty=10000

\setlength{\parskip}{1ex}
\setlength{\parindent}{0ex}
\setstretch{1.5}


\begin{document}

\title{Social security US\thanks{Lorenzo Zecca, Bonn University. Email: \href{mailto:s78lzecc@uni-bonn.de}{\nolinkurl{s78lzecc [at] uni-bonn [dot] de}}.}}

\author{Lorenzo Zecca}

\date{
    31st March 2023
}

\maketitle


\begin{abstract}
    This project replicates the study of Conesa and Krueger (1999): we consider a discrete time overlapping generations model, 
    where the economy is populated by a continuum with given mass 
    growing at a constant rate $n$ of ex-ante identical individuals.
    We compare two steady states: 
    one in which the government runs a social security system, financed 
    through taxes on labor; and another one, where the there is no public 
    pension system, and eranings from labor are not taxed. However, we will cover only 
    the comparion between the two steady states, neglecting the transition dynamics analysis.
\end{abstract}

\clearpage


\section{Numerical results} % (fold)
The figures below show life-cycle profiles of labor supply, assets, consumption, and earnings in the two steady states of
our model. However, the images are obtained setting only 2 iterations. A proper analisys would indeed need around 30 to get reasonable results.
This project uses only 2 to speed up the code and show that the project itself is running.

\begin{table}[!htb]
\input{../bld/python/results/table_results.tex}
\end{table}
\section{Aggregate variables' profile: Initial vs. Final Steady State}

\begin{figure}

    \centering
    \includegraphics[width=0.85\textwidth]{../bld/python/figures/steady_states/savings_by_age.png}
    \caption{\emph{Python:} Savings by age. Larger in the final ss: without social security, agents have to accumulate more wealth.}
    \label{fig:python-K}
\end{figure}

\begin{figure}
    
    \centering
    \includegraphics[width=0.85\textwidth]{../bld/python/figures/steady_states/labor_by_age.png}
    \caption{\emph{Python:} Effective labor supply by age.}
    \label{fig:python-L}

\end{figure}

\begin{figure}

    \centering
    \includegraphics[width=0.85\textwidth]{../bld/python/figures/steady_states/earnings_by_age.png}
    \caption{\emph{Python:} Earnings by age. Larger in the final ss: agents work more and at higher wages.}
    \label{fig:python-E}

\end{figure}


\begin{figure}

    \centering
    \includegraphics[width=0.85\textwidth]{../bld/python/figures/steady_states/consumption_by_age.png}
    \caption{\emph{Python:} Consumption by age.}
    \label{fig:python-C}

\end{figure}

\printbibliography

\end{document}
